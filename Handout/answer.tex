\documentclass[11pt]{article} 
\usepackage{bm}
\usepackage{cmap}
\usepackage{ctex}
\usepackage{cite}
\usepackage{color}
\usepackage{float}
\usepackage{xeCJK}
\usepackage{amsthm}
\usepackage{amsmath}
\usepackage{amssymb}
\usepackage{setspace}
\usepackage{geometry}
\usepackage{hyperref}
\usepackage{enumerate}
\usepackage{indentfirst}
\usepackage{yfonts}
\usepackage{pdfpages}

\usepackage[cache=false]{minted}

%代码高亮
\geometry{margin=1in}

\newcommand{\cppcode}[1]
{
  \inputminted[mathescape,
  tabsize=4,
  linenos,
  framesep=2mm,
  breakaftergroup=true,
  breakautoindent=true,
  breakbytoken=true,
  breaklines=true,
  fontsize=\small
  ]{cpp}{Source/#1}
}

\newcommand{\javacode}[1]
{
  \inputminted[mathescape,
  tabsize=4,
  linenos,
  framesep=2mm,
  breakaftergroup=true,
  breakautoindent=true,
  breakbytoken=true,
  breaklines=true,
  fontsize=\small
  ]{java}{Source/#1}
}

\newcommand{\emacscode}[1]
{
  \inputminted[mathescape,
  tabsize=4,
  linenos,
  % frame=single,
  framesep=2mm,
  breakaftergroup=true,
  breakautoindent=true,
  breakbytoken=true,
  breaklines=true,
  fontsize=\small
  ]{emacs}{Source/#1}
}

\newcommand{\verilogcode}[1]
{
  \inputminted[mathescape,
  tabsize=4,
  linenos,
  % frame=single,
  framesep=2mm,
  breakaftergroup=true,
  breakautoindent=true,
  breakbytoken=true,
  breaklines=true,
  fontsize=\small
  ]{verilog}{#1}
}

\title{\Huge Computer Architecture Homework of Lmxyy\thanks{https://github.com/lmxyy/Computer-Architecture-Task-1}}
\author{李沐阳 516021910346 \\ \emph{Shanghai Jiao Tong University}}
\date{\today}
\begin{document}
\maketitle
\newpage
\tableofcontents
\newpage
\section{Task 1}
\subsection{Subtask 1}
\verilogcode{adder.v}
\newpage
\subsection{Subtask 2}
\verilogcode{adder2.v}
\subsection{Subtask 3}
答:由于超前16位进位加法器中,进位不存在数据依赖性,即进位不需要依赖前一位的结果,可以从输入信号中直接得出,从而可以进行并行计算。他优化了暴力加法器中,数据转运所需要的时间。

\section{Task 2}
\subsection{Subtask 1}
\verilogcode{fault.v}
\subsection{Subtask 2}
\verilogcode{fault2.v}
\newpage
\subsection{Subtask 3}
答:从软件思路讲,这时:
\begin{itemize}
\item wire对应于连续赋值,如assign;
\item reg对应于过程赋值,如always,initial。
\end{itemize}

从电路角度考虑,这时:
\begin{itemize}
\item wire型的变量综合出来一般是一根导线;
\item reg变量在always块中有两种情况:
  \begin{enumerate}
  \item always后的敏感表中是(a or b or c)形式的,也就是不带时钟边沿的,综合出来还是组合逻辑;
  \item always后的敏感表中是(posedge clk)形式的,也就是带边沿的,综合出来一般是时序逻辑,会包含触发器(Flip-Flop)。
  \end{enumerate}
\end{itemize}
\subsection{Subtask 4}
答:搭配assign时,或要变量连续赋值时,reg不能够当成wire;否则可以将reg当成wire用。
\end{document}